% Double spacing, if you want it.
\def\dsp{\def\baselinestretch{2.0}\large\normalsize}
\dsp

% If the Grad. Division insists that the first paragraph of a section
% be indented (like the others), then include this line:
% \usepackage{indentfirst}

%% \maketitle
\approvalpage
\copyrightpage


\begin{abstract}
Biosensors are everywhere these days. But what can their data reveal about us -
about the inner parts of our minds, our moods, our emotions, identities? My
dissertation will explore \textit{people's beliefs} about answers to this
question, how these beliefs affect and arise from their interactions with
digital sensors, and existing beliefs about the body. The cases in this
dissertation surface an unstable boundary between sensing bodies and sensing
minds, and proposes this boundary as a site for studying the impact that these
devices will play in the future.
\end{abstract}

%% As biosensors creep into smart watches, bands, and ingestibles, they
%% will build increasingly high resolution models of bodies in space. Their ability
%% to divine not just what these bodies do, but what they think and feel, presents
%% an under-explored avenue for understanding and imagining the role these
%% technologies will play in everyday life.


\begin{frontmatter}

\begin{dedication}
\null\vfil
\begin{center}

%% # \begin{flushright}
%% # for the most beautiful mom the world.
%% # we had an amazing, wonderful life together, a life as beautiful and precious as you are.
%% # dad and i carry that life forward everyday, because of the unconditonal love you gave to us,
%% # the greatest, most important gift anyone could ever receive.
%% # the gift you gave to me and dad.
%% # i will always take care of him.
%% # until we see you again.
%% # I love you forever.

%% # also, I have been shaving.
%% # \end{flushright}
To Mom\\\vspace{12pt}
I've been shaving (mostly).
Thank you for everything.
I love you forever.
\end{center}
\vfil\null
\end{dedication}


\tableofcontents
\clearpage
\listoffigures
\clearpage
\listoftables

\begin{acknowledgements}
Thanks to John Chuang for being a mentor in every sense of the word, for taking
me on in this ancient model of apprenticeship, and showing me how the work is
done.

During my time as a PhD student, John put together BioSENSE, a group that made
my dissertation possible. A few BioSENSors in particular helped me immensely:
Coye Cheshire, with whom I did much of the work in this dissertation, and fellow
students (and extended cohort members) Richmond Wong, and Noura Howell.

Along the way, Paul Duguid taught me a great deal about what good scholarship
looks like, and I even got to coauthor a paper with him.

Above all, I would like to thank my mom and dad, who believed in the importance
of my work when I didn't.

I would also like to thank Chihei Hatakayama for the peaceful music.
\end{acknowledgements}

%% # Machines will someday read the mind.
%% # I know. But they will.
%% # ``We'' will find a way to make them,
%% # and ``we'' will use any tools at our disposal to do so.

%% # The primary tool we will use is our own interpretation, and theories
%% # of the mind, projecting it onto a physical reality
%% # that is less definite than ``we'' would sometimes like to imagine.

%% # Who is this ``we''?
%% # This dissertation looks primarily (though not exclusively) at software professionals
%% # in the San Francisco Bay Area.

%% # _etc..._

%% # Nick Merrill
%% # Berkeley, California
\end{frontmatter}
